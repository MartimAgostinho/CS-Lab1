\section{Introduction}

Analog-to-digital converters (ADCs) are fundamental components in modern electronic systems. Among various ADC architectures, the Successive Approximation Register (SAR) ADC has emerged as a dominant choice for medium-resolution, energy-efficient applications. This report investigates the behavioral modeling and performance analysis of a high-performance n-bit asynchronous SAR ADC based on the design presented in the IEEE Access paper titled \textit{"A 6.94-fJ/Conversion-Step 12-bit 100-MS/s Asynchronous SAR ADC Exploiting Split-CDAC in 65-nm CMOS"} (2021)\textsuperscript{\cite{paper}}. The primary objective was to develop a high-level model that accurately captures the ADC's operation while incorporating practical non-idealities such as capacitor mismatch, parasitic capacitances, and comparator offset voltage.

The referenced ADC achieves remarkable energy efficiency of $6.94 \si{\femto \joule}$ per conversion-step through several key innovations. First, it employs a split-capacitor digital-to-analog converter (split-CDAC) architecture that strategically divides the capacitor array into a 7-bit most-significant-bit (MSB) segment and a 4-bit least-significant-bit (LSB) segment, connected through a bridge capacitor. This approach significantly reduces the total number of unit capacitors required compared to conventional binary-weighted arrays, thereby minimizing area and power consumption. Second, the ADC utilizes asynchronous control logic that eliminates the need for a high-frequency system clock, instead relying on self-timed comparators and delay-based bit-cycling to optimize both speed and power efficiency. Third, the design incorporates a programmable dummy capacitor in the LSB array to compensate for parasitic-induced nonlinearity, along with custom-designed metal-oxide-metal (MOM) unit capacitors that improve matching characteristics.

This report analyses the ADC's operation through mathematical modeling and computational simulations. Beginning with a derivation of the charge redistribution equations for the split-CDAC's behavior, we develop model that accurately reproduces the ADC's conversion process. The model includes realistic component variations, specifically accounting for Gaussian-distributed capacitor mismatches with $\sigma = 0.11\%$ and comparator input-referred offsets with $\sigma = 10$ mV. Using this model, we evaluate the ADC's linearity performance by computing integral and differential nonlinearity (INL/DNL) metrics and analysing the effective number of bits (ENOB) through fast Fourier transform (FFT) analysis of the output spectrum.

To assess the design's robustness against manufacturing variations, we performed Monte Carlo simulations consisting of $20000$ iterations with randomized component mismatches. This statistical analysis provides valuable insights into the ADC's parametric yield and guides design trade-offs, such as the bridge capacitor value. Furthermore, we compare our simulation results with the results from the original paper.

The significance of this work lies in its demonstration of how high-level behavioral modeling can effectively predict ADC performance prior to fabrication, enabling rapid design exploration and optimization. 

The remainder of this report is organized as follows: Section~\ref{sec:adc_in_study} reviews the ADC architecture and operating principles, Section~\ref{sec:circuit_analysis} details the mathematical modeling approach, Section~\ref{sec:modeling} focus on the implementation of the Python script used to model the ADC, Section~\ref{sec:simulation} presents simulation results and analysis, Section~\ref{sec:implementation} estimates silicon implementation factors, Section~\ref{sec:results} presents the results analysis and comparison with the paper mentioned, and Section~\ref{sec:conclusion} concludes with key findings and their implications for ADC design. 

