\section{Circuit Implementation}
\label{sec:implementation}

In the pursuit of achieving the minimum area and power for $10.5$ bit linearity, the first step is to find the smallest capacitor area that gives a low enough variation for the requested linearity. Figure \ref{fig:Csize2sigma}, shows the area size to capacitance relation, now finding the minimum area is matter of finding the biggest variation. 

\begin{figure}[H]

    \centering
    \includegraphics*[width=0.5\textwidth]{Images/Csize2sigma.png}
    \caption{Capacitor area to capacitance variation.\textsuperscript{\cite{paper}}}

    \label{fig:Csize2sigma}
\end{figure}

Using Figure's \ref{fig:Csize2sigma} linear regression Equation \ref{eq:AreaSigma} is deduced.

\begin{equation}
    \text{area} = \left(\frac{0.408}{\sigma_\% - 0.01}\right)^2 = \SI{0.170}{\micro\meter}^2 
    \label{eq:AreaSigma}
\end{equation}

\subsection{Variation}

To ensure a high yield in production, a scenario where 96\% of ADC units would meet or exceed the 10.5-bit linearity specification was targeted. 

With $\sigma = 1\%$, 96.55\% of results had a linearity of $10.5$ or bigger, Table \ref{tab:ResultsLin105} shows the characteristics for this variation.

\begin{table}[H]
    \centering
    \caption{Monte Carlo (M.C.) $\sigma = 1\%$.}
    \begin{tabularx}{\textwidth}{
      >{\centering\arraybackslash}X 
      >{\centering\arraybackslash}X 
      >{\centering\arraybackslash}X 
      >{\centering\arraybackslash}X 
      >{\centering\arraybackslash}X
      >{\centering\arraybackslash}X
    }
    \toprule
    \textbf{Simulation} & \textbf{Case} & \textbf{$V_{\text{LSB}}~[\si{\milli\volt}]$} & \textbf{Linearity} & \textbf{SNR [\si{\decibel}}] & \textbf{ENOB} \\

        \midrule
        % Here: span 3 rows in first column
        \multirow{3}{*}{
            \makecell[c]{%
                M.C. \\
                All Caps.\\
                $\sigma=1\%$
            }%
        } 

        & Min & 0.4849 &   9.7896 &  53.4049  &  8.5789  \\\cline{2-6}
        & Typ & 0.4850 &   11.1765 &  55.0371  &  8.7534 \\\cline{2-6}
        & Max & 0.4906 &   12.8408 &  54.4554  &  8.8500 \\

      \bottomrule
    \end{tabularx}
    \label{tab:ResultsLin105}
\end{table}

Analysing Figure \ref{fig:Lin_Dist_s01}, it is clear to see that the majority of cases meet the specification. 

\begin{figure}[H]

    \centering
    \includegraphics*[width=0.8\textwidth]{Images/Lin_s01.png}
    \caption{ADC's Linearity distribution for $\sigma = 1\%$.}

    \label{fig:Lin_Dist_s01}
\end{figure}

\subsection{Power Consumption}

By definition Power is $P = I\cdot V$, because $I = Q / \Delta t$ therefore for the circuit power Equation \ref{eq:CircPower} was used.

\begin{equation}
    P = \frac{V^2 \cdot C }{\Delta t} = V^2 \cdot C \cdot f
    \label{eq:CircPower}
\end{equation}

For the worst case $V = V_{r}$, frequency $f = \SI{100}{\mega\hertz}$ and the capacitance can be calculated by summing all MSB capacitors and the $C_B$ paralell with the sum of the LSB capacitores, $C = 128.7778\cdot C_u$ therefore:

\begin{equation}
    P = 2\cdot V_{r}^2 \cdot 128.778\cdot C_u \cdot 10^8
\end{equation}
