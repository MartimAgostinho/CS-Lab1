\section{Circuit Analysis}
\label{sec:circuit_analysis}

In this section, we will analyze the circuit of the ADC, focusing on the CDAC.

\subsection{Digital to Analog Converter Analysis}

A SAR ADC fundamentally implements a binary-search conversion by using a capacitive DAC to generate successive approximations of the input voltage. This makes the DAC the most important block to analise in this ADC. 

For theoretical analysis, we collapse the differential front-end into a pseudo-single-ended equivalent, making the analysis simpler, the analyzed circuit is represented in Figure \ref{fig:DAC_Circ}. 

It is important to note that the DAC code is divided in Most Significant Bits (MSB) and Least Significant Bits (LSB), in the circuit this division is provided by the capacitor $C_B$.

\begin{figure}[H]
    \centering
    \includegraphics*[scale = 0.35]{Images/DacCirc.png}
    \caption{Simplified DAC circuit.}
    \label{fig:DAC_Circ}
\end{figure}

Each DAC capacitor's bottom plate is alternately connected to either ground (0 V) or the reference voltage $V_{ref}$. During conversion, that switching injects or removes charge proportional to the capacitor's value and the switch state (0 or 1). Hence, in order to model the circuit behavior can be modeled as a voltage source with value $b_n\cdot C_n$. 


For an arbitrary phase $\phi_n$, the circuit is represented by Figure \ref{fig:P1_Circ}. If this phase is the sampling phase, $V_x = V_{in}$ and all capacitors are connected to ground, except the hybrid capacitors as explained at section \ref{sec:CDAC}.

\begin{figure}[H]

    \centering
    \includegraphics*[width=0.8\textwidth]{Images/DACCircPn.png}
    \caption{Circuit at phase $\phi_n$.}

    \label{fig:P1_Circ}
\end{figure}


Analysing the sum of charges at the node $V_x$ during phase $\phi_n$, the Equation \ref{eq:ChargeP1} is obtained.

\begin{equation}
    Q_{\phi_n} = (V_x^{\phi_n}-V_y^{\phi_n})\cdot C_B + V_x^{\phi_n}\cdot \underbrace{ \left ( C_B + \sum_{i=0}^{I}C_i \right )}_{C_{MT}}-V_r\cdot \left ( \sum_{i=0}^{I} C_i\cdot b_i \right )^{\phi_n}
    \label{eq:ChargeP1}
\end{equation}

Similarly for phase $\phi_{n+1}$, the Equation \ref{eq:ChargeP2}.

\begin{equation}
    Q_{\phi_{n+1}} = (V_x^{\phi_{n+1}}-V_y^{\phi_{n+1}})\cdot C_B + V_x^{\phi_{n+1}}\cdot C_{MT}-V_r\cdot \left ( \sum_{i=0}^{I} C_i\cdot b_i \right )^{\phi_{n+1}}
    \label{eq:ChargeP2}
\end{equation}

Therefore, with $Q_{\phi_{n}}=Q_{\phi_{n+1}}$:

\begin{equation}
    \begin{split}
        V_x^{\phi_{n+1}} &= V_x^{\phi_{n}} +  \frac{ V_r\overbrace{\sum_{i}\left[ (C_i\cdot b_i)^{\phi_{n+1}} - (C_i\cdot b_i)^{\phi_{n}}\right]}^{\Delta C_i}+C_B\cdot \left(V_y^{\phi_{n+1}}-V_y^{\phi_{n}}\right)}{C_{MT}} \\
        V_x^{\phi_{n+1}} &= V_x^{\phi_{n}} + V_r \frac{ V_r \Delta C_i +C_B\cdot \left(V_y^{\phi_{n+1}}-V_y^{\phi_{n}}\right)}{C_{MT}}
    \end{split}
    \label{eq:VxPn}
\end{equation}

Assessing Equation \ref{eq:VxPn}, Capacitors that remain connected to the same source between phases, do not contribute to the net change in charge. The capacitors that change state between phases are represented as $\Delta C_i$. This means that if a capacitor changes from ground to $V_{ref}$, $\Delta C_i = C_i$ but in the case where it changes from $V_{ref}$ to ground,  $\Delta C_i = -C_i$, this will e important for the hybrid capacitors.

To discover the value of $V_y$, the process is similar. 

\begin{equation}
    Q_{\phi_{n}} = V_y^{\phi_n}\underbrace{\left( \sum_{j=0}^{J}C_j \right)}_{C_{LT}} + (V_y^{\phi_{n}}-V_x^{\phi_{n}})C_B-V_r\cdot \left ( \sum_{j=0}^{J} C_j\cdot b_j \right )^{\phi_n}
    \label{eq:VyP1}
\end{equation}

\begin{equation}
    Q_{\phi_{n+1}} = V_y^{\phi_{n+1}}\underbrace{\left( \sum_{j=0}^{J}C_j \right)}_{C_{LT}} + (V_y^{\phi_{n+1}}-V_x^{\phi_{n+1}})C_B-V_r\cdot \left ( \sum_{j=0}^{J} C_j\cdot b_j \right )^{\phi_n+1}
    \label{eq:VyP2}
\end{equation}

Solving Equation \ref{eq:VyP1} and Equation \ref{eq:VyP2}, $V_y^{\phi_{n+1}}-V_y^{\phi_{n+1}}$ is obtained, Equation \ref{eq:Vy}.

\begin{equation}
    V_y^{\phi_{n+1}}-V_y^{\phi_{n+1}} = C_B\cdot (V_x^{\phi_{n+1}}-V_x^{\phi_n}) + V_r\Delta C_j
    \label{eq:Vy}
\end{equation}

Substituting, $V_y^{\phi_{n+1}}-V_y^{\phi_{n+1}}$ in Equation \ref{eq:VxPn} and simplifying, Equation \ref{eq:DeltaVx} is obtained.

\begin{equation}
    \boxed{V_{x}^{\phi_{n+1}}=V_{x}^{\phi_n}+V_{r}\frac{   {\Delta}C_{Mi} (C_{B} + \sigma_{LC} )+C_{B} {\Delta}C_{Li}}{C_{B}\left( \sigma_{LC} + \sigma_{MC}\right) + \sigma_{LC} \sigma_{MC}} = V_x^{\phi_n} + \Delta V_x^{\phi_{n+1}} (b_n)}
    \label{eq:DeltaVx}
\end{equation}

Evaluating Equation \ref{eq:DeltaVx}, a conclusion can be made that $V_x^{\phi_{n+1}}$ only depends on the previous value and how the capacitors change independently. This will be helpful for implementation. 

% \textcolor{red}{Analyze the capacitor array circuits (shown in Fig. 1 in the paper) to obtain the
% expression of the capacitor array voltages (Vxp and Vxn) as a function of vin, the control bits and the capacitors values. Use the charge conservation principle in
% your analysis. Suggestion: first analyze the MSB sub-array separately.Explain how you obtained the expressions and present the expressions in a readable format.}
