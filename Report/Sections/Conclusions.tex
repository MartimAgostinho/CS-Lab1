\section{Conclusion}
\label{sec:conclusion}

This report presented the behavioral modeling and analysis of a 12-bit asynchronous SAR ADC leveraging a split-CDAC architecture in 65-nm CMOS, inspired by the design from Li et al. (2021). The ADC achieved remarkable energy efficiency through three key innovations: a split-CDAC reducing unit capacitors by 97\%, a hybrid capacitor configuration stabilizing common-mode voltage, and asynchronous logic eliminating high-frequency clock dependencies.

\subsection{Key Points}

\subsubsection{Model Validation}

The derived charge-redistribution equations and Python-based simulations accurately replicated the ADCs operation, including non-idealities like capacitor mismatch ($\sigma=0.11\%$) and comparator offset ($\sigma=10 mV$). Monte Carlo analysis (20,000 iterations) confirmed robustness, with typical $ENOB = XXX \si{bits}$ and $SNR = 73.97 \si{\dB}$ under ideal conditions.

The split-CDAC architecture significantly reduced area and power, with 144 unit capacitors vs. 4095 in a conventional binary-weighted array.

\subsubsection{Performance Trade-offs}

Bridge Capacitor ($C_B$): Simulations showed $CB=\frac{16}{15}$ offered superior linearity ($ENOB =11.99 bits$) but $CB=2$ was more area-efficient with minimal performance loss.

Comparator Offset: A $10 mV$ offset introduced missing codes near zero-crossing, degrading ENOB to $8.5 bits$ in worst-case scenarios.

\subsubsection{FoM Analysis}

Using $P= XX mW$, $ENOB=XX bits$, and $fs=100MS/s$, the calculated FoM was $XX fJ/conversion-step$, close to the paper's $6.94 fJ/step$.

\subsubsection{Comparison with Referenced Paper}

    The project's FoM aligns with state-of-the-art SAR ADCs but highlights the impact of modeling assumptions. The referenced paper's digital correction (DEC) technique, not modeled here, likely enhanced linearity and FoM.

\subsubsection{Future Work}

Implement Digital Calibration: Integrate DEC to mitigate capacitor mismatch and offset errors, improving ENOB.

\subsubsection{Summary}
The project successfully modeled a 12-bit asynchronous SAR ADC with a split-CDAC architecture, achieving a FoM of XXX fJ/conv.-step. The behavioral modeling approach provided valuable insights into the ADC's operation and performance trade-offs, paving the way for future enhancements through digital calibration techniques.

This work underscores the efficacy of behavioral modeling in predicting ADC performance and guiding design trade-offs. 

The project successfully replicated the core principles of the referenced ADC, demonstrating the potential for high-efficiency, medium-resolution converters in modern CMOS processes. 