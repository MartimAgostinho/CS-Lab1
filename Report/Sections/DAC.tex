\section{ARRANJAR TITULO}

Para analisar o circuit primeiro dividir porque é diferencial. e analisar primeiro o DacCirc

Explicar que o codigo ]e dividido em dois codigos
Falar dos split caps

\begin{figure}[H]
    \centering
    \includegraphics*[scale = 0.35]{Images/DacCirc.png}
    \caption{Simplified DAC circuit}
    \label{fig:DAC_Circ}
\end{figure}

explicar fases

\subsection{Phase 1}

\begin{figure}[H]

    \centering
    \begin{subfigure}{0.4\textwidth}
        \includegraphics*[scale = 0.17]{Images/DACCircP1.png}
        \caption{Complete Circuit}
        \label{fig:first}
    \end{subfigure}
    \hfill
    \begin{subfigure}{0.4\textwidth}
        \includegraphics*[scale = 0.18]{Images/DACP1Ceq.png}
        \caption{Simplified Circuit}
        \label{fig:second}
    \end{subfigure}
    
    \caption{Phase 1 circuit}
    \label{fig:P1_Circ}
\end{figure}

Where:
\begin{equation}
    C_{eq} = C_B // \overbrace{\left( C_{0'}+\sum_{i=0}^{Ln} C_i\right )}^{S_{LC}}
\end{equation}

\begin{equation}
    \begin{split}
        Q_{\phi 1} &= V_x^{\phi_1}\cdot C_{M0}+ V_x^{\phi_1}\cdot  \sum_i C_{Mai} + (V_x^{\phi_1} - V_{ref})\cdot  \sum_i C_{Mbi}+ V_x^{\phi_1}\cdot C_{eq} \\
        &= V_x^{\phi_1} \left( C_{eq} +\underbrace{ C_{M0} +   \sum_i C_{Mai}+C_{Mbi}}_{S_{MC}} \right ) - V_{ref}\cdot \sum_i C_{Mbi}
    \end{split}
    \label{eq:ChargeP1}
\end{equation}


\subsection{Phase 2}

\textcolor{red}{Dizer que pode ser analisado sem split}

For the second phase the capacitors are either connected to ground or $V_{ref}$ METER REF AO CIRCUIT COM OS NOS VX e VY. Therefore:

\begin{equation}
    Q_{Mn} = (V_x^{\phi_1} - V_{ref}\cdot b_{Mn})\cdot C_{Mn}
    \label{eq:ChargeMn}
\end{equation}

The equation \ref{eq:ChargeMn} represents the charge stored on the capacitor $C_{Mn}$, where the subscript $Mn$ denotes the n-th bit capacitor within the MSB code block, and the term $b_{Mn}$ is the value associated with the mentioned bit.

\begin{equation}
    \begin{split}   
        Q_{\phi_2} &= (V_x^{\phi_2} - V_{ref}\cdot b_{Mn})\cdot C_{Mn} + (V_x^{\phi_2} - V_y^{\phi_2})\cdot C_B \Leftrightarrow \\
       \Leftrightarrow Q_{\phi_2} &= V_x^{\phi_2}\left[C_B + \sum_{n=0}^{Mn}C_{Mn}\right]-V_{y}^{\phi_2}\cdot C_B - V_{ref}\cdot\underbrace{\sum_{n=0}^{Mn}C_{Mn} b_{Mn}}_{S_{MB}(code)}
    \end{split}
    \label{eq:ChargeP2}
\end{equation}

Now $V_y$ can easily be analysed with superposition:


\begin{figure}[H]
    \centering
    \includegraphics*[scale = 0.35]{Images/VySuperposition.png}
    \caption{$V_y$ superposition circuit ARRANJAR OUTRO TITULO}
    \label{fig:VySuperposition}
\end{figure}

\textcolor{red}{EXPLICAR}

It is important to note that there is a special case for the source $V_x$. 

\begin{equation}
    V_{yn}^{\phi_2} = \frac{C_{Ln}}{\displaystyle C_B + C_{B0'}+\sum_i C_{Li}}\cdot b_{Ln}\cdot V_{ref}
\end{equation}

\begin{equation}
    \begin{split}
        V_y^{\phi_2} &= V_x^{\phi_2}\frac{C_B}{C_B + C_{0'}+\sum_i C_{Li}} + \sum V_{yn}^{\phi_2} =\\
        &= \frac{ V_x^{\phi_2}\cdot C_B + V_{ref}\cdot \overbrace{\sum_i b_{Li}\cdot C_{Li}}^{S_{LB}}}{ C_B + C_{0'}+\sum_i C_{Li}}
    \end{split}
    \label{eq:Vy}
\end{equation}

Solving the equations ... \textcolor{red}{meter ponte e palha}

python:

\begin{equation}
    \boxed{ V_x(code) = V_{i}  + V_{ref}\frac{ S_{MB}(code)\cdot(C_B+S_{LC})+S_{LB}(code)\cdot C_B  - S_{LC}\cdot\sum C_{Mbi}}{C_{B} ( S_{LC} + S_{MC}) + S_{LC} S_{MC}}}
    \label{eq:VxFinal}
\end{equation}
